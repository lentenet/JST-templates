%!TEX TS-program = xelatex
%!TEX encoding = UTF-8 Unicode
%
% Journal for Scripture & Theology Article Style Template -- http://scripturetheology.ca/
%     LaTeX preamble for use with companion LaTeX and MultiMarkdown templates
%     See the Readme.md file at https://github.com/scripturetheology/JST-templates
% Created by Daniel Driver (2011) -- http://danieldriver.com/
% License: http://creativecommons.org/licenses/by-nc/3.0/
%
\documentclass[letterpaper,twoside]{article}
\usepackage[textwidth=340pt,textheight=550pt,marginratio=340:550,includeheadfoot]{geometry}
% 550 / 340 = 1.618 ≈ φ
%
% Other useful packages
\usepackage[usenames,dvipsnames]{color}
\usepackage{xcolor}			% Allow for color (annotations)
\usepackage{graphicx}		% To enable including graphics in pdf's
\usepackage{booktabs}		% Better tables
\usepackage{tabulary}		% Support longer table cells
\usepackage{rotating}		% use {sidewaystables} if desired
\usepackage{fancyvrb}		% Allow \verbatim et al. in footnotes
\VerbatimFootnotes
%
% Typography
\frenchspacing
\usepackage{fontspec} 
\defaultfontfeatures{Mapping=tex-text}
\setromanfont[Ligatures={Common}]{Skolar PE}
\setmonofont[Scale=0.8]{Menlo}
\newfontfamily{\sbl}[Script=Hebrew]{SBLHebrew}
% SBL Hebrew can be downloaded at http://www.sbl-site.org/educational/BiblicalFonts_SBLHebrew.aspx
%
% \usepackage{bidi} (for bidirectional typesetting) must be called near the end of the preamble;
% see documentation at http://mirror.ctan.org/macros/latex/contrib/bidi/bidi.pdf
%
% Swashes, arrows and dots:
% all of which need the commercial Skolar pan-European font -- http://www.rosettatype.com/Skolar
\newcommand{\rare}[1]{\fontspec[Ligatures={Rare}]{Skolar PE}\selectfont #1}
\newcommand{\arrows}[1]{\fontspec[Variant=6]{Skolar PE}\selectfont #1}
\newcommand{\uppersc}[1]{\fontspec[Letters={UppercaseSmallCaps}]{Skolar PE}\selectfont #1}
\newcommand{\point}{\unskip~{\arrows T}~\ignorespaces}
\newcommand{\jst}{\emph{{\rare jst}.st}}
\newcommand{\step}{\unskip\nobreak\enspace{\arrows -\textbackslash}\enspace\ignorespaces}
%
% Ragged marginal notes
\let\oldmarginpar\marginpar
\renewcommand\marginpar[1]{\oldmarginpar[\raggedleft\footnotesize %{\arrows v|-.}\\\smallskip
 #1]{\raggedright\footnotesize %{\arrows .-|v}\\\smallskip
 #1}}
%
% Journal metadata
\def\journalname{\emph{Journal for Scripture \& Theology}}
\def\jstdoi{10.5827}
\def\issn{1925-9506}
\def\jstcopyright{Copyright \copyright~\jstyear\ JST and the author, released under a \href{http://creativecommons.org/licenses/by-nc/3.0/}{\textsc{cc by-nc 3.0} license}.}
%
% Article metadata
\def\keywords{}
\def\datesubmitted{\today}
\def\dateaccepted{TBD}
\def\datepublished{TBD}
\def\jstyear{\year}
\def\jstvol{1}
\def\jstiss{1}
\def\firstpage{TK}
%
% Other variables & placeholders
\def\myauthor{Please list A. N. Author}
\def\institution{TK. The author should provide an institution and/or a place. Titles, URLs and mailing addresses are optional. An email address is optional, too, but strongly encouraged.}
\def\authoremail{tk.author@institution.edu}
\def\authornote{}
\def\shorttitle{TK short title for the running header}
%
% Variables specific to reviews
\def\bookauthor{TK author(s) under review}
\def\booktitle{TK full title of the book under review}
\def\bookpublication{TK publication details, e.g.: Oxford Studies in Historical Theology. Oxford/New York: Oxford University Press, 2010. Pp. xiv + 222. Hardcover. \$74.00.}
\def\shortauthor{\bookauthor}
